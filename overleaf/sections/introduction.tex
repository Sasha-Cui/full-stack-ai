\documentclass[class=article,crop=false]{standalone}
\usepackage{Draft,SashaMacros}
\begin{document}
\section{Introduction}
  Large language models (LLMs) and their surrounding ecosystem have evolved rapidly, creating a landscape where the best researchers are expected to navigate multiple layers of the AI stack.  From downloading sharded model checkpoints to deploying distributed reinforcement learning pipelines, a modern AI researcher must operate as both an engineer and a scientist. 

  \paragraph{Problem}
    The traditional division of labor, where some focus only on mathematical modeling and others only on systems, has given way to a full-stack expectation: the ability to move fluidly between data handling, model training, evaluation, deployment, and interpretability.  However, we cannot find resources that can introduce AI researchers to understand the entirety of the technical stack.  

  \paragraph{Contribution}
    In the Fall 2025 semester at Yale University, we organised a \textbf{Becoming Full-Stack AI Researchers} working group.  This tutorial paper and its accompanying GitHub repository emerged out of the working group.  Our goal is not just to introduce individual frameworks in isolation, but to show how they fit together into an end-to-end research workflow.  We emphasize hands-on, reproducible, and extensible practices. Each section is written by contributors who prepared minimal working examples (MWEs), demos, slides, and live presentations, thereby capturing both the technical details and the practical challenges encountered by new users.
    
  \paragraph{Application}
    This tutorial can be used as the starting point of a one-semester course on modern AI development as well as for self-study of interested learners who have basic familiarity with Python.  Our intended audience includes graduate students, postdoctoral researchers, and faculties who want to broaden their AI engineering capabilities. By the end of this tutorial, readers should be able to replicate common workflows, adapt them to their own research problems, and understand the broader design space of tools and frameworks that constitute modern AI development.

  \paragraph{Acknowlegements}
    This course is generously supported by the Wu Tsai Institute at Yale. 

\end{document}