\documentclass[class=article,crop=false]{standalone}
\usepackage{Draft,SashaMacros}
\begin{document}
\section{Ray: A Distributed Computing Framework}
  \subsection{Motivation and Background}
    As compute and memory requirements of LLMs keep increasing, single-server settings can no longer accommodate their development, training, and serving needs.  
    Ray~\cite{raypaper} addresses this challenge by enabling fast and easy application scaling to distributed environments. 
    It provides scalable libraries to support the complete life-cycle of ML applications, pythonic primitives for intuitive development, and integration with existing tools and infrastructure like Kubernetes, AWS, and Azure.
    
    
    At its core, Ray utilizes three main abstractions:
    \begin{itemize}
      \item \textbf{Tasks (remote functions):} A Python function decorated with \texttt{@ray.remote}. 
      Each invocation launches a stateless parallel computation executed on any available CPU or GPU.
      \item \textbf{Actors:} A Python class decorated with \texttt{@ray.remote} becomes an actor.  Unlike tasks, actors are \emph{stateful}---they live on a worker process and can preserve variables across multiple method calls.
      \item \textbf{Worker Node:} Under the hood, Ray maintains a pool of worker nodes, which consist of multiple physical processes. Each node hosts a \emph{raylet} process, which manages the node's shared resources, and one or more worker processes responsible for task submission and execution.
    \end{itemize}
    
    Ray's cluster hosts the head node, which contains the \emph{Global Control Store}(GCS), a server managing cluster-level metadata and operations like scheduling.
    For fault tolerance, Ray introduced an ownership model, which can be combined with lineage reconstruction to recover lost objects in the event of a node's failure. 
    These components contribute to Ray's core principles for API simplicity and generality, while enabling system performance and reliability.
    More information about Ray's architecture can be found in the designated architecture whitepaper~\cite{rayarchitecture}.

    

    
    \subsection{Historical Context and Ecosystem}
    Researchers at UC Berkeley’s RISELab introduced Ray at 2018~\cite{raypaper} and it has since
    evolved into a widely adopted ecosystem for scaling AI and Python workloads. 
    Today, Ray is maintained by Anyscale and supports a range of libraries for specialized tasks:
    \begin{itemize}
      \item \textbf{Ray RLlib}: High-Performance RL
      \item \textbf{Ray Serve}: Scalable \& Programmable Model Serving.
      \item \textbf{Ray Tune}: Scalable Hyperparameter Tuning.
      \item \textbf{Ray Train}: Distributed Model Training
      \item \textbf{Ray Data}: Distributed Data Processing
    \end{itemize}
  
  \subsection{Core Programming Model and Syntax}
    The minimal Ray program utilizes standard Python. 
    To enable Ray, we import and initialize it:
    \begin{verbatim}
    import ray
    ray.init()  # connect to cluster or start locally
    \end{verbatim}
    
    \paragraph{Remote functions (tasks).} Decorate a Python function with \texttt{@ray.remote} to execute it as a parallel task:
    \begin{verbatim}
    @ray.remote
    def square(x):
      return x * x
    
    futures = [square.remote(i) for i in range(4)]
    print(ray.get(futures))  # [0, 1, 4, 9]
    \end{verbatim}
    
    \paragraph{Actors.} Use actors to encapsulate state across multiple method calls. An actor
    is a class decorated with \texttt{@ray.remote} and runs persistently on
    a worker process:
    \begin{verbatim}
    @ray.remote
    class Counter:
      def __init__(self):
        self.value = 0
      def increment(self):
        self.value += 1
        return self.value
    
    counter = Counter.remote()
    print(ray.get(counter.increment.remote()))  # 1
    \end{verbatim}
    
    \paragraph{Workers.} Behind the scenes, Ray launches a pool of worker processes (Python
    interpreters). 
    Tasks and actors are scheduled onto these workers by the
    Ray runtime. 
    While you rarely interact with workers directly, you can
    control resource allocation (e.g., CPUs, GPUs) when defining tasks or actors:
    \begin{verbatim}
    @ray.remote(num_cpus=2)
    def heavy_task(x):
      return x ** 2
    \end{verbatim}
    Here, Ray will schedule \texttt{heavy\_task} only on workers with at least two available CPUs.

    \subsection{Strengths, Weaknesses, and Integration}
    
    \paragraph{Strengths.} The Python-first design of Ray lowers the barrier for distributed computing. 
    Its modular libraries (Tune, RLlib, Serve) cover end-to-end machine learning workflows. 
    It integrates naturally with PyTorch, TensorFlow, and JAX.
    
    \paragraph{Weaknesses.} Ray introduces runtime overhead for small-scale tasks, and cluster management may require careful tuning for performance.  
    It is best suited for medium to large-scale AI applications in distributed environments rather than fine-grained parallelism in single-server settings.
    
    \subsection{Installation and Getting Started}
    Ray can be installed with:
    \begin{verbatim}
    pip install ray
    \end{verbatim}
    
    Launching Ray locally requires no configuration. To connect to a
    multi-node cluster, one can start the head node with
    \verb|ray start --head| and join worker nodes with
    \verb|ray start --address='...'|.
    
    Ray can also be combined with orchestration systems such as Kubernetes
    or Slurm for deployment on HPC or cloud environments.
\end{document}